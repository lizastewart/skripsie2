\chapter{Results} \label{results}
The results of the MCMC analysis as well as the MAP optimisation is visualised in the section below.
\section{Continuous output}
To assess the progress of the MCMC analysis while iterating, graphs similar to those in Figure~\ref{fig:10} were continuously generated. 
In Figure \ref{fig:10}, the output of 2~000 iterations were extracted to allow sufficient detail to be visible.
Figure \ref{fig:10a} was important to keep track of the number of different samples assessed until an acceptable sample was found.
If a large number of samples were consistently needed, it could indicate that a bad starting point was chosen or that there is an error within the posterior distribution calculation.
The connection between Figures \ref{fig:10b} and \ref{fig:10c} can be seen in that when there is a dip in the difference between $\pi^*(x)$ and $\pi^*(x_{\text{ref}})$, there is a peak in the acceptance ratio.
When the graph in Figure \ref{fig:10b} is at 0, the sampled values are the same as the Eurocode reference values.
When this graph is below 0, the values sampled for that iteration produced a less likely result than the reference values.
This influenced the burn-in period chosen to ensure that all the samples considered were more accurate than the Eurocode values.
As the probability function used is not normalised, it can produce posterior probabilities larger than 1; therefore, the difference between two posterior probabilities can be larger than 1.
This also means that the higher the difference between $\pi^*(x)$ and $\pi^*(x_{\text{ref}})$, the more accurate the $x$ value is.

\begin{figure}[H]
	\centering
	
	\begin{subfigure}[b]{\textwidth}
	\centering	
	\includegraphics[width = \textwidth]{figures/figure10a.jpg}
	\caption{Number of samples drawn before an acceptable sample was found per iteration}
	\label{fig:10a}
	\end{subfigure}{}
	
	\begin{subfigure}[b]{\textwidth}
	\centering	
	\includegraphics[width = \textwidth]{figures/figure10b.jpg}
	\caption{Difference between the accepted $x$ posterior and the reference posterior for each iteration}
	\label{fig:10b}
	\end{subfigure}{}
	
	\begin{subfigure}[b]{\textwidth}
	\centering	
	\includegraphics[width = \textwidth]{figures/figure10c.jpg}
	\caption{Value of acceptance ratio at each iteration}
	\label{fig:10c}
	\end{subfigure}{}
	
	\caption{Graphs from continuous output}
	\label{fig:10}
\end{figure}



\section{Resulting \texorpdfstring{$\kappa$}{TEXT}-values}

The thermal conductivity ($\kappa$-values) at key temperatures integral to this project are shown in Table~\ref{krestab}. 
As can be seen in Figure \ref{kresult_euro_fig}, there was a drastic difference in the thermal diffusivity at 1 200 \textdegree C and between 0\textdegree C and 200\textdegree C.
The large difference at 1 200\textdegree C can be explained by how the model was created; at that temperature, the majority of the heat is transferred through thermal radiation.
The model did not separately model thermal radiation, but instead incorporated the radiation into an equivalent conduction.
Between 0\textdegree C and 200\textdegree C, the modelling of the evaporation was moderately successful as a spike can be seen at 100 \textdegree C.  

\begin{table} [H]
\centering
\caption{Posterior thermal conductivity in W/(m$\ $K)}
\label{krestab}
	\begin{tabular}{ r r r r }
	\toprule
	 \textdegree C & EURO & MAP & MCMC\\
	\midrule
	0   & 0.12&0.062 &0.12\\
	60  &0.12& 0.189 &0.278\\
	100 &0.12& 0.204 &0.247\\
 	140 &0.12& 0.096 &0.242\\
	200 & 0.15& 0.236 &0.243\\
	350 & 0.07& 0.117  &0.096\\
	500 & 0.09& 0.039 &0.120\\
	800 & 0.35& 0.265 &0.234\\
	1200& 1.5& 7.943 &7.467\\
	\bottomrule	
	\end{tabular}
	
\end{table}
\begin{figure}[H]
\centering	
	\includegraphics[width=\textwidth]{figures/kvalues_all.png}
	\caption{Resulting $\kappa$-values compared to Eurocode standard values}
	\label{kresult_euro_fig}
\end{figure}
 
 The uncertainty associated with the $\kappa$-values is indicated in Figure \ref{errorbars}.
 Error bars indicating one standard deviation have been added.
 
 \begin{figure}[H]
\centering	
	\includegraphics[width=\textwidth]{figures/KerrorSTD2.png}
	\caption{Markov chain Monte Carlo $\kappa$-values with standard deviation error bars }
	\label{errorbars}
\end{figure}
 
 For each temperature, there will be a different distribution of samples. 
 In Figure \ref{fig:scatdata}, the distribution of all the samples were plotted in a scatter plot.
 The resulting $\kappa$-values from the MCMC algorithm is plotted in red and the maximum a posteriori is plotted in blue.
 The asymmetrical distribution  is clear in the data as the maximum a posteriori is not in the centre of each distribution.

%TODO
From the difference between the histogram maximum and the maximum found by the Nelder-Mead optimisation at 140\textdegree C (Figure \ref{fig:histo}), it is clear that many more iterations of the MCMC algorithm is needed.
In contrast, at 60\textdegree C the maximum of the histogram and the MAP is the same, indicating that the distribution at that specific temperature was sufficiently explored.
An estimated 500 000 iterations would be needed.
As the current iterations already took four days, an additional 14 weeks would have been needed to generate the necessary additional samples.
This would have been impractical to achieve during a single semester, as these iterations needed the full computational capacity of the available machine and prevented any other usage thereof.

\begin{figure}[H]

\centering
	\begin{subfigure}[b]{0.45\linewidth}
	\centering
	\includegraphics[width = \linewidth]{figures/histograph/histo1.png}
	\end{subfigure}
	\begin{subfigure}[b]{0.45\linewidth}
	\centering
	\includegraphics[width = \linewidth]{figures/histograph/histo3.png}
	\end{subfigure}
	\centering
	\begin{subfigure}[b]{0.2\linewidth}
	\centering
	\includegraphics[width = \linewidth]{figures/histograph/legend.png}
	\end{subfigure}
	\caption{Distribution of samples at 60\textdegree C and 140\textdegree C with MCMC and MAP results indicated}
	\label{fig:histo}
\end{figure}

%\begin{figure}[b]
\label{histofig}
\centering
	\begin{subfigure}{}
	\centering
	\includegraphics[width = 0.45\linewidth]{figures/histograph/histo1.png}
	\end{subfigure}
	\begin{subfigure}{}
	\centering
	\includegraphics[width = 0.45\linewidth]{figures/histograph/histo2.png}
	\end{subfigure}
	\begin{subfigure}{}
	\centering
	\includegraphics[width = 0.45\linewidth]{figures/histograph/histo3.png}
	\end{subfigure}
	\begin{subfigure}{}
	\centering
	\includegraphics[width = 0.45\linewidth]{figures/histograph/histo4.png}
	\end{subfigure}
	\begin{subfigure}{}
	\centering
	\includegraphics[width = 0.45\linewidth]{figures/histograph/histo5.png}
	\end{subfigure}
	\begin{subfigure}{}
	\centering
	\includegraphics[width = 0.45\linewidth]{figures/histograph/histo6.png}
	\end{subfigure}
	\begin{subfigure}{}
	\centering
	\includegraphics[width = 0.45\linewidth]{figures/histograph/histo7.png}
	\end{subfigure}
	\begin{subfigure}{}
	\centering
	\includegraphics[width = 0.45\linewidth]{figures/histograph/histo8.png}
	\end{subfigure}
	\centering
	\begin{subfigure}{}
	\includegraphics[width = 0.3\linewidth]{figures/histograph/legend.png}
	\end{subfigure}
	\caption[short]{Distribution of samples at different temperatures with MCMC and MAP results indicated}
\end{figure}
\input{fig_scatterdata.tex}

\subsection{Evaluation of error}
Despite the uncertainties connected to the problem, a quantifiable measure of error is still required to allow unbiased assessment of the success of the algorithm and results.
The obtained $\kappa$-values will be used to rerun the model and obtain the new modelled temperature for each depth. 
For each depth, the error will be quantified using the norm of the relative error at the seven different depths, denoted as $i$.
The relative error percentages will be calculated according to  Equation \ref{relerror}. 
The new modelled output errors are compared to the Eurocode $\kappa$-value output errors to assess if the methods used produced a more accurate representation of the data.
The error at different depths are summarised in Table \ref{errortab} and the lowest error at that depth is highlighted in green. 
At a depth of 0 mm, none of the errors are highlighted, as this temperature is independent of the $\kappa$-values.
This error measurement can be used to asses the error in the model or assumptions made to obtain the model.
It is important to note that this error does not only quantify the error in the thermal conductivity, but that any errors due to model assumptions are also encompassed.

\begin{equation}\label{relerror}
\epsilon _i = \frac{\norm{\hat{x}_i-x_i }_2}{\norm{x_i}_2}\times 100 \quad \quad i= 1, ... ,7
\end{equation}


%\begin{table}[H] 
%\centering
%\caption{Summary of error between models and data}
%\label{errortab}
%	\begin{tabular}{ r r r r }
%	\toprule
%	Depth(mm) & EURO & MAP & MCMC\\
% \midrule
%0& 0.231	& 0.231	& 0.231\\
%16.5& 0.179	& \cellcolor{green!20}0.143	& 0.154\\
%33& 0.334	& \cellcolor{green!20}0.179	& 0.189\\
%49.5& 0.608	& 0.377	& \cellcolor{green!20}0.369\\
%66& 0.552	& 0.384	&\cellcolor{green!20} 0.309\\
%82.5& 0.490	&\cellcolor{green!20} 0.348	& 0.366\\
%99& 0.182	& 0.182	&\cellcolor{green!20} 0.181\\
%\midrule
%RMS & 0.404	& 0.280	& \cellcolor{green!20}0.270\\
%\bottomrule	
%	\end{tabular}	
%\end{table}


\begin{table}[H] 
\centering
\caption{Summary of error percentage between models and data}
\label{errortab}
	\begin{tabular}{ r r r r }
	\toprule
	Depth(mm) & EURO & MAP & MCMC\\
\midrule
0	&	23.11&	23.11&	23.11\\
16.5	&	17.94&	\cellcolor{green!20}14.27&	15.35\\
33	&	33.40&	\cellcolor{green!20}17.89&	18.94\\
49.5	&	60.78&	37.69&	\cellcolor{green!20}36.93\\
66	&	55.18&	38.39&	\cellcolor{green!20}30.92\\
82.5	&	48.99&	\cellcolor{green!20}34.76&	36.60\\
99	&	18.18&	18.16&	\cellcolor{green!20}18.12\\
\midrule
RMSE	&	40.43&	28.01&	\cellcolor{green!20}27.04\\
\bottomrule	
	\end{tabular}	
\end{table}
%\begin{equation}\label{rmseq}
%RMS_j = \sqrt{\frac{1}{n}\sum_{i}  }\quad \quad j= 1, ... ,7
%\end{equation}

The root mean square error (RMSE) for each method was also calculated. 
There is a definite improvement in the accuracy of the model using the values obtained from the MCMC analysis compared to using the Eurocode values.



\subsection{Thermal diffusivity}
The main intent of this project was to determine the thermal diffusivity of SA-Pine. 
In Table~\ref{diffrestab}, the resulting thermal diffusivity is shown.
The thermal diffusivity was calculated using the information in Table~\ref{cptab} and Equation~\ref{alphaeq}.

\begin{table}[H]
\centering
	\caption{Resulting thermal diffusivity ($\alpha \ten{-4}$) in m/s$^2$ }
	 \label{diffrestab}
	\begin{tabular}{ r r r r }
	\toprule
	Temp \textdegree C & EURO & MAP & MCMC\\
	\midrule
	0&   $1.64$&	$0.85$&	$1.64$\\
	60&  $1.64$&	$2.58$&	$3.79$\\
	100& $1.64$&	$2.78$&	$3.37$\\\
	140& $1.64$&	$1.31$&	$3.30$\\
	200& $2.05$&	$3.21$&	$3.32$\\
	350& $0.96$&	$1.60$&	$1.31$\\
	500& $1.23$&	$0.53$&	$1.64$\\
	800& $4.78$&	$3.62$&	$3.19$\\
	1200& $25.00$&	$108.00$&	$102.00$\\
	\bottomrule	
	\end{tabular}
	
\end{table}


\begin{sidewaysfigure}
	
	\centering
	\includegraphics[width=\linewidth,]{figures/finalgraph.png}
	\caption{Graph of measured data compared to model output}
	\label{final_graph}
\end{sidewaysfigure}





