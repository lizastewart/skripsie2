\chapter{Technical Foundation} \label{tech}
\section{Finite Element Model}
Finite element methods (or finite element analysis) is used when the behaviour of an element cannot be accurately depicted by a simple mathematical equation. 
	\subsection{History/Origin}
	The finite element method (FEM) used today is the sum of decades of research. 
	In an article by \citeauthor{Gupta:1996} they discuss the five main contributors to the finite element method. 
	According to \citet{Gupta:1996} the idea behind the finite element method was initially explored in the \citeyear{Courant:1943} article by \citeauthor{Courant:1943}. 
	Courant acknowledges the complex nature of mathematical problems in his first paragraph by stating: "Mathematics is an indivisible organism uniting theoretical contemplation and active application."
	He goes on to discuss the variational method
	
	
	
	\subsection{method of FEM}
	A larger element is broken into smaller elements. 
	Assumptions made on the smaller scale have a lesser effect on the final answer than the same assumptions made on a large scale would have had.


	\subsection{heat eq}	 
	 In its simplest form, the one-dimensional heat diffusion equation is a partial differential equation \ref{heat_eq} dependant on the temperature and thickness of the element. 
	The heat diffusion equation is based on Fourier's Law...(TODO)
	
	
	\begin{equation}
	\label{heat_eq}
		q = -k \frac{dT}{dx}
	\end{equation}

\section{Bayes' theorem of inverse problems}
%	Statistical and Computational Inverse problems by Kaipio and Somersalo Chapter 3
% 	The Bayesian approach to Inverse Problems Dashti and Stuart
	The method of statistical inversion is dependant on a fundamental understanding of the Bayes' theorem of inverse problems. 
	The student obtained this understanding through studying Chapter 3 of statistical and Computational Inverse problems by \citet{Kaipo:2005}, further referred to merely as Kaipio. 
	There are four principles of Statistical inversion that is essential to the thorough understanding of these models. 
	Firstly, it is the principle that any variable in the model needs to be modelled as a random variable. 
	This randomness is based on the extent of information that is available. 
	To ensure that the extent of knowledge is accurately portrayed in the model, the extent of knowledge will be coded into the probability distributions assigned to the different variables. 
	Finally, it needs to be understood that the solution of a statistical inversion is a posterior probability distribution.
	A generalized equation of Bayes' theorem can be seen in \ref{bayes_eq} taken from Kaipio. 
	
	\begin{equation}
	\label{bayes_eq}
	\pi_{\text{post}}(x) = \pi(x|y_{\text{observed}}) = \frac{\pi_{\text{pr}}(x) \pi(y_{\text{observed}}|x)}{\pi (y_{\text{observed}})}	
	\end{equation}

\section{Markov Chain Monte Carlo} \label{MCMCdet}
Markov Chain Monte Carlo (MCMC) is a method of integration. This will be used to determine the mean of the $\kappa$-values at specific temperatures. 
	Markov Chain Monte Carlo is a method that was created by combining the concept of Monte Carlo sampling  and a Markov Chain. 
	To fully understand MCMC, the methods that it was created from need to be further investigated.
	\subsection{Markov Chains}
	%Relevant literature to cite:	
		
		
	\subsection{Monte Carlo Integration}
	Monte Carlo integration is used to evaluate a probability distributuion. 
	The evaluation is done by drawing a collection of random numbers from the distribution.
	These numbers are then used as the sample and a sample mean is taken.
	The arithmetic sample mean can be used to approximate the population mean in accordance with the law of large numbers \citep{Gilks:1996}.
	All of the random samples are not generally accepted.
	Here the acceptance criteria comes into play.
	There are multiple options for how a posterior is deemed acceptable, these are elaborated on in the book  \textit{Monte Carlo Statistical Methods} by  \author{Robert:2004}. 
	
%	\begin{equation}
%	\label{acceptcrit}
%	
%	\end{equation}
\subsection{Metropolis-Hastings Algorithm}

The Metropolis-Hastings algorithm is a simulation method based on the MCMC principles.
