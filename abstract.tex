%*** THE ABSTRACT PAGE ************************************

\begin{abstract}
%  What the objectives of the study were (the central question);
  Stochastic inversion methods such as Markov chain Monte Carlo and optimisation methods were used to determine the thermal diffusivity of SA Pine.
  Thermal diffusivity can be easily calculated from thermal conductivity, and conductivity can be more directly solved through modelling; the focus was therefore moved to the calculation of thermal conductivity.
%Brief statement of what was done (Methods);
A finite element model was constructed to model a $100$mm element exposed to the ISO 846 Fire curve.
The model was compared to data previously obtained by S. van der Westhuyzen; this data was also used as the observed quantity in the formulation of the Bayes' model.
Using Bayes inversion, a posteriori probability distribution consisting of a likelihood probability and a prior probability was constructed.
The finite element model was essential to the calculation of the likelihood probability. 
The prior probability was based off the thermal conductivity values from EN 1995:1-2-2004.
%Brief statement of what was found (Results);
New thermal conductivity values were estimated through Markov chain Monte Carlo inversion and the Maximum a Posteriori was found though optimisation.
The Markov chain Monte Carlo and maximum a posteriori thermal conductivity were not equal, supporting the hypothesis that the distribution is not normal.
%The FEM model using the new $\kappa$-values was closer to the measured temperatures.
The model was rerun with the new thermal conductivity estimates and the models were compared to the the measured data.
The new estimated conductivity values resulted in more accurate modelling of the temperature in SA Pine when exposed to fire.
From the new conductivity estimates combined with known values of density and specific heat capacity, the thermal diffusivity could be successfully calculated.
%Brief statement of what was concluded (Discussion);

\end{abstract}

\begin{abstract}[Afrikaans]
Stogastiese inversie metodes soos Markovketting Monte Carlo en optimalisering metodes was gebruik om die termiese diffusiwiteit van Suid-Afrikaanse dennehout te bepaal.
Termiese diffusiwiteit kan maklik uitgewerk word as die termiese geleidingsvermo{\"e} bekend is.
Termiese geleidingsvermo{\"e} kan meer direk bepaal word deur modellering, en daarom het die fokus verskuif na die bepaling van die termiese geleidingsvermo{\"e}.
'n Eindige element model was saamgestel om a $100$mm element wat blootgestel is aan die ISO 846 vuur kurwe te modelleer.
Di{\'e} model was vergelyk met data wat vooraf deur S. van der Westhuyzen gemeet is; hierdie data was gebruik as die waarneembare hoeveelheid in die opstelling van die Bayes model.
Deur gebruik van Bayes inversie was 'n posterior verspreiding wat bestaan uit 'n waarskynlikheid verspreiding en 'n voorafgaande verspreiding bepaal.  % haha gatkantverspreiding
'n Eindige element model was benodig om die waarskynlikheid verspreiding uit te werk.
Die voorafgaande verspreiding was gebaseer op die termiese geleidingsvermo{\"e} waardes in EN 1995:1-2-2004.
Nuwe waardes vir die termiese geleidingsvermo{\"e} was bepaal deur Markovketting Monte Carlo inversie en die maximum a posteriori was gevind deur optimalisering.
Die gemiddeld wat bevind was deur die Markovketting Monte Carlo en deur die maximum a posteriori was nie gelyk nie, wat die hipotese dat die verspreiding nie normaalverspreid is nie ondersteun.
Nuwe temperatuur uitslae was met behulp van die model en die nuwe termiese geleidingsvermo{\"e} bepaal.
Die nuwe termiese geleidingsvermo{\"e} waardes het gelei tot meer akkurate modellering van die temperatuur in Suid-Afrikaanse dennehout wat blootgestel is aan vuur. 
Die termiese diffusiwiteit kon bereken word vanaf die nuwe geleidingsvermo{\"e} waardes, die bekende digtheid, en die spesifieke hittekapasiteit.

\end{abstract}
\endinput
