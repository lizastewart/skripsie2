%*** THE ABSTRACT PAGE ************************************

\begin{abstract}
%  What the objectives of the study were (the central question);
  Stochastic inversion methods such as Markov Chain Monte-Carlo and optimisation methods were used to determine the thermal diffusivity of SA-Pine.
  As thermal diffusivity can be easily calculated from thermal conductivity and conductivity can be more directly solved through modelling the focus was moved to the calculation of thermal conductivity.
%Brief statement of what was done (Methods);
A finite element model was constructed to model a $100$mm element exposed to the ISO 846 Fire curve.
The model was compared to data previously obtained by S. van der Westhuyzen, this data was also used as the observed quantity in the formulation of the Bayes' model.
Using Bayes' theorem of inverse problems a posteriori probability distribution consisting of a likelihood probability and a prior probability was constructed.
The finite element model was essential to the calculation of the likelihood probability. 
The prior probability was based off the thermal conductivity values from EN 1995:1-2-2004.
%Brief statement of what was found (Results);
New thermal conductivity values were estimated through Markov Chain Monte-Carlo inversion and the Maximum a Posteriori was found though optimisation.
The Markov Chain Monte-Carlo and Maximum a Posteriori thermal conductivity were not equal, supporting the hypothesis that the distribution is not normal.
%The FEM model using the new $\kappa$-values was closer to the measured temperatures.
The model was rerun with the new thermal conductivity estimates and the models were compared to the the measured data.
The new estimated conductivity values resulted in more accurate modelling of the temperature in SA Pine exposed to fire.
From the new conductivity estimates and known values of density and specific heat capacity the thermal diffusivity could be successfully calculated.
%Brief statement of what was concluded (Discussion);

\end{abstract}

\endinput
