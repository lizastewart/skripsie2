\chapter{Discussion} \label{discussion}
%TODO

In Figure \ref{final_graph}, the difference in output from running the model with the new $\kappa$-values, Eurocode $\kappa$ and the measured data can be seen. 
The attempt at modelling the water evaporation at 100\textdegree C was not as successful as initially hoped as the model did not have the same flattening at 100\textdegree C as the measured data.
The improved accuracy is clear in Table \ref{errortab}, as the RMSE for the model using MCMC values is smaller than the error incurred by the same model using the Eurocode thermal conductivity values.
There are instances where the error from the model using the MAP $\kappa$-values is smaller than the model using the MCMC $\kappa$-values; this should not discredit the usage and value of the Markov chain Monte Carlo algorithm as the increased accuracy from the MAP values is only marginal.
It is also important to remember that the MCMC algorithm did not necessarily find the population mean due to insufficient iterations.
This problem can be simply solved by running more iterations.
The error at 49.5mm is visible in Figure~\ref{final_graph} after 3~000 seconds, where the measured temperature is significantly higher than the Eurocode model output.
The increase of error in the centre of the sample is due to the assumption that the timber stays intact.
This assumption oversimplifies the problem; as the timber starts charring, pieces of the timber fall off, exposing an internal thermocouple directly to the fire.


%    Summarize the important findings of your observations.
The $\kappa$-values obtained from the MAP and MCMC analysis give a more accurate model output than those from the Eurocode.
This increased accuracy serves as a proof of concept that MCMC analysis can be used to determine more accurate fire ratings and specifically fire resistance. 

Fire resistance is measured in minutes, indicating the amount of time from the start of the fire until specific conditions are met. 
The main conditions taken into account are accounted for by the REI marker system \citep{rei:2021}. 
In this acronym, `R' indicates the load-bearing capacity of the structural element and its associated time indicates when the element can no longer carry the design load.
`E' refers to the integrity of the element and the time indicates how long after ignition the fire penetrates the element.
`I' indicates insulation ability; a limit is set to the temperature of the non-heated side and the time corresponding with that rating refers to when that temperature is exceeded on the non-heated side.

With further development of analytical determination of thermal conductivity and applied finite element models, the data obtained from fire test could be used in a model to determine the REI markers of differently sized elements.
%    For each result, describe the patterns, principles, relationships your results show. Explain how your results relate to expectations and to references cited. Explain any agreements, contradictions, or exceptions. Describe what additional research might resolve contradictions or explain exceptions.

%    Suggest the theoretical implications of your results. Extend your findings to other situations or other species. Give the big picture: do your findings help us understand a broader topic?
The accuracy of the Markov chain Monte Carlo analysis can be increased by modelling the standard deviation of both the temperatures and the $\kappa$-values as random values and solving for them as well. 
This would make the analysis more heuristic and decrease the dependence of the results on assumptions made by the researcher. 
Additionally, the program could have been run for a longer time to ensure more samples. and thereby a more accurate approximation of the population mean. 
Unfortunately, the main limitation in this project was computational time.

It is known that the probability of the thermal conductivity is not symmetrically distributed and that the mean is a more accurate description of the data than the maximum a posteriori.
However, the maximum a posterioiri can be used to evaluate if sufficient samples were generated.




\cite{somasund:2016} also used the Metropolis-Hastings algorithm with Markov chain Monte Carlo (MH-MCMC) sampling to inversely solve heat transfer equations.
They additionally compared the MH-MCMC sampling technique to both the parallel tempering sampling technique and the evolutionary Monte Carlo techniques.
They found that the standard MH-MCMC sampling technique is accurate enough when a relatively small standard deviation is used, but that the accuracy drastically decreases with higher standard deviations.
This is further proof that also solving for the standard deviation would increase the accuracy of the MH-MCMC analysis.
They also found that the parallel tempering and evolutionary Monte Carlo techniques were far more accurate even with higher standard deviations.
%In retrospect there should have been more cleaning of the measured data before analysis was done. Inaccuracies arose due to the first 90 sec of the measurements being taken before the furnace was turned on and the thermocouples still measuring after the fire died down.
%These measurement discrepancies between model and measurement should have been better taken into account and excluded form modelling. 

The research presented in this report can be expanded and applied to the measured data available for eucalyptus. 
Applying the same algorithm to different data sets and obtaining accurate modelling from both data sets would be confirmation that the algorithm is accurate enough to be further explored for usage in practice.



