\chapter{Derivation}\label{appder}
\begin{equation}
	%%\label{heateq1}
	q_{,x}-f = 0  \text{  ...(1)} \quad\quad\quad\quad q = -\kappa u_{,x} \text{  ...(2)} 
	\end{equation}
	Integrating Equation \ref{appheateq1} (1) over the length of the element and introducing a weighting function $w(x)$ we obtain \ref{appheateq2}. 
	Since the derivative of $w(0)$ is known and $q_{,x}$ is unknown. 
	The first term in \ref{appheateq2} is integrated by parts. 
	After the integration by parts and substituting $q$ with \ref{appheateq1} (2), Equation \ref{appheateq3} is created.
	\begin{equation}
	%%\label{heateq2}
	\int_{x=0}^L wq_{,x}dx - \int_{x=0}^L wfdx = 0
	\end{equation}
	\begin{equation}
	%%\label{heateq3}
	\int_{x=0}^L wku_{,x}dx + \int_{x=0}^L wfdx - \left.wq\right|_0^L = 0
	\end{equation}
	In Equation \ref{appheateq3} the $u$ and $w$ need to be defined.
	Assuming $u \approx u^h$ and $w \approx w^h$ and using the basis function ($N^A$), Equation \ref{appbasisfunc_eq} is obtained.
	\begin{equation}
	\label{appbasisfunc_eq}
	\begin{aligned}
	u_e^h = \sum_{B} N^B d^B  \quad &;\quad w_e^h = \sum_{A} N^A c^A\\
	u_{e,x}^h = \sum_{B} N_{,x}^B d^B \quad &;\quad w_{e,x}^h=\sum_A N_{,x}^A c^A\\
	\end{aligned}
	\end{equation}		
	Substituting the $u$ and $w$ functions back, we obtain the Galerkin weak form shown in Equation \ref{appheateq4}.

	\begin{equation}
	\label{appheateq4}
	\begin{aligned}
	\sum_e \int_{\Omega_e} w_{e,x}^h k  u_{e,x}^h dx &+ \sum_e \int_{\Omega_e} w_e^h f dx - w(L)q_L + w(0)q_0 &= 0\\
	\int_{\Omega_e} \sum_A \sum_B N_{,x}^A c^A k N_{,x}^B d^B dx  &+ \int_{\Omega_e} \sum_{A} N^A c^A f dx - \sum_{A\in A_N} c^A q^A &= 0\\
	\end{aligned}
	\end{equation}
	\begin{equation}
	%%\label{heateq5}
	\vec{c}_e^T \vec{K}_e^{AB} \vec{d}_e + \vec{c}_e^T \vec{F}_e^f -\vec{c}_e^T \vec{F}_e^q
	\end{equation}
	where
	\begin{equation*}
	\begin{aligned}
	K_e^{AB} &= \int_{\Omega_e}N_{,x}^AN_{,x}^B k dx\\
	F_e^{Af} &= \int_{\Omega_e}N_{,x}^A f dx\\
	F_e^{Aq} &= \begin{cases} q(x_N) &\text{ for } x \in \Gamma_N\\0 &\text{ for other }\\
	\end{cases}
	\end{aligned}
	\end{equation*}
	
	
	\begin{equation} \label{appheatconeq}       
	       \begin{aligned}
	       q_{\text{adv}}= \nu &\rho c_p \Delta T \quad = \quad h \Delta T\\	      
	       \nu =& \text{velocity m/s}\\
	       \rho =& \text{density of air}\\
	       c_{\rho} =& \text{heat capacity of air}\\
          \end{aligned}
	\end{equation}
	       
	\begin{equation} \label{appradeq}
	       \begin{aligned}
	       q_{\text{rad}} = \varepsilon&\sigma \phi \left(T_f^4 - T_S^4\right)\\
	       \varepsilon =& \text{emissivity}\\
	       \sigma =& \text{Stefan-Boltzmann} \\&5.670 e^-8 [W/(m^2K^4)]\\
	       \phi =& \text{view factor};1 \quad \text{here}\\
	       \end{aligned}
	\end{equation}
	Given that $T_S$ and $T_F$ are known a new equivalent heat flux value can be calculated as in Equation \ref{femboundeq1}.
	\begin{equation} \label{appfemboundeq1}
	q_{\text{con}}^{\text{equiv}} = \kappa^{\text{equiv}} \frac{\Delta T}{\Delta L} = q_{\text{rad}} + q_{\text{adv}}\\
	\end{equation}
	Due to the clear relationship between heat flux and thermal diffusivity the equivalent diffusivity($\kappa^{\text{equiv}}$) can be calculated as shown below in Equation \ref{femboundeq2}. 
	\begin{equation}\label{appfemboundeq2}
	\begin{aligned}
	\kappa^{\text{equiv}} &= \frac{\left[q_{\text{rad}} + q_{\text{adv}} \right]}{\Delta T}\\
	&\quad\\
	&= \frac{\epsilon \sigma \phi \Delta \left(T^4\right)}{\Delta T / \Delta L } + h \Delta L\\
	&\quad\\
	&= \frac{\epsilon \sigma \Delta \left(T^4\right)}{\Delta T   } + h\\
	\end{aligned}
	\end{equation}
	
Diffusion Equation: 
	\begin{equation}\label{appfemdiffeq1}
	q_{,x} - f = \frac{\partial Q}{\partial t} = c_{\rho}\frac{\partial u}{\partial t}\quad\text{  ...(1)} \quad\quad\quad q = -\kappa \frac{\partial u}{\partial x}\quad\text{  ...(2)}
\end{equation}
Substituting Equation \ref{appfemdiffeq1}(2) into \ref{appfemdiffeq1} and taking the derivative as indicated ($q_{,x}$) gives Equation \ref{appfemdiffeq2}. 
As previously disccussed heat conduction ($\alpha$) is heat diffusion ($\kappa$)  divided by specific heat ($c_p$).


\begin{equation}\label{appfemdiffeq2}
\begin{aligned}
\therefore -\kappa \frac{\partial^2 u}{\partial x^2} - f &= c_{\rho}\frac{\partial u}{\partial t} \rightarrow f=0:\\
\frac{\partial^2 u}{\partial x^2} &= - \frac{c_{\rho}}{\kappa} \frac{\partial u}{\partial t} \quad \\
&\text{ or } \\
\frac{\partial u}{\partial t} &= -\alpha \frac{\partial^2 u}{\partial x^2}
\end{aligned}
\end{equation}


Let $c_p = \lambda$ 

\begin{equation}\label{appfemdiffeq3}
\therefore -\kappa u_{,xx} - \lambda u_{,t} = f
\end{equation}


Then:

\begin{equation}\label{appfemdiffeq4}
\int_0^L w q_{,x} \text{d}x - \int_0^L w \lambda u_{,t} \text{d}x - \int_0^L w f \text{d}x = 0
\end{equation}


Similar to what was done in \ref{appbasisfunc_eq} a special approximation is made to obtain Equation \ref{appfemdiffeq5}.

\begin{equation}\label{appfemdiffeq5}
u \approx u^h \rightarrow u_e^h = \sum_{A}N^A d^A \quad;\quad w_e^h = \sum_{A}N^A L^A
\end{equation}


\begin{equation}\label{appfemdiffeq6}
\rightarrow \sum_e \int_{\Omega^e} w_{e,x}^h \kappa u_{e,x}^h \text{d}x_e + \sum_e \int_{\Omega^e} w_e^h \lambda u_{e,t}^h \text{d}x_e + \sum_e \int_{\Omega^e} w_e^h f \text{d}x_e - \sum_{e \in \epsilon_A} w q_N = 0
\end{equation}

\begin{equation*}
*_1: \int_{\Omega^e} \sum_A \sum_B N_{,x}^A c^A \kappa N_{,x}^B d^B \text{d}x = \sum_A\sum_B \left[ \int N_{,x}^A N_{,x}^B \kappa \text{d}x  \right]d^B = \vec{c}_e^T \vec{\kappa}_e \vec{d}_e
\end{equation*}


\begin{equation*}
*_2: \int_{\Omega_e} \sum_A \sum_B N^A c^A \lambda N^B \dot{d}^B \text{d}x = \sum\sum c^A \left[ \int N^A N^B \lambda \text{d}x \right] d^B = \vec{c}_e^T \vec{M}_e \vec{\dot{d}}_e
\end{equation*}


\begin{equation*}
*_3: \int_{\Omega^e} \sum_A N^A c^A f \text{d}x = \sum c \int_B^A N f \text{d}x = \vec{c}_e^T \vec{F}_e^B
\end{equation*}


\begin{equation*}
*_4: \sum_{A \in \mathcal{A}} \kappa^A q_N^A = \vec{c}_e^T \vec{F}_e^q
\end{equation*}


From all the above equations the below matrix formulation could be assembled:
%considere renamin to somehing with assembled or matrix in the name
\begin{equation}\label{appfemdiffeq7}
\vec{c}^T \vec{\kappa} \vec{d} + \vec{c}^T \vec{M} \vec{\dot{d}} = \vec{c}^T \vec{F}
\end{equation}




\begin{equation}
\beign{aligned}
(1) \text{ Set }& \vec{d}_0 \text{equal to the known conditions at time 0}\\
&  \vec{v}_0 = \vec{0} \\
\text{Also set } \alpha \text{as a parameter between 0 and 1} \\
 &\Delta t \text{is the timestep}
\end{aligned}
\end{equation}

Following the v-form Time integration in Hughes Chapter 8 , page 460, define $\vec{d}_{n+1$ as in Equation \ref{appeqdn1}.
\begin{equation}\label{appeqdn1}
\vec{d}_{n+1} = \vec{d}_n + (1-\alpha) \Delta \vec{v}_n + \alpha \Delta t \vec{v}_{n+1}
\end{equation}


\begin{equation}
(\vec{M} + \alpha\Delta t \vec{\kappa})\vec{v}_{n+1} = \vec{F}_{n+1} - \vec{\kappa} \tilde{\vec{d}}_{n+1}
\end{equation}


If $\vec{M}$ and $\vec{\kappa}$ independent of time and temperature,it is sufficient to say:

\begin{equation}
\vec{v}_{n+1} = (\vec{M}+\alpha\Delta t \vec{\kappa})^{-1} (\vec{F}_{n+1} - \vec{\kappa} \vec{\tilde{d}}_{n+1})
\end{equation}

%Page 5 
If $\kappa$ and $\lambda$ is independent of position:
\begin{equation}
\begin{aligned}
	\kappa_e^{AB} &= \int_0^\ell N_{,x}^A N_{,x}^B \kappa \text{d}x \\
&= \kappa \int_0^\ell N_{,x}^A N_{,x}^B \text{d}x \\
&= \kappa \int_{-1}^{+1} N_{,\xi}^A N_{,\xi}^B \xi_{,x}^2 \left|J\right|  d\xi \\
&=\kappa \frac{2}{\ell} \int_{-1}^{+1} N_{,\xi}^A N_{,\xi}^B \text{d}\xi\\
&= \pm \frac{\kappa}{\ell} = \begin{cases}
+\frac{1}{2} & \text{if } A = B \\
-\frac{1}{2} & \text{if } A \ne B
\end{cases}
\end{aligned}
\end{equation}

\begin{equation}
\therefore \vec{\kappa}_e = \frac{\kappa}{\ell} \begin{bmatrix} +1 & -1 \\ -1 & +1 \end{bmatrix}
\end{equation}	

\begin{equation}
M_e^{AB} = \int_0^\ell N^A N^B \lambda \text{d}x = \lambda \int_0^\ell N^A N^B \text{d}x = \frac{\lambda \ell}{2} \int_{-1}^{+1} N^A N^B \text{d}x
\end{equation}
\begin{equation}
N^1 N^1 = \frac{1}{4}(1-\xi)^2 ; N^2 N^2 = \frac{1}{4}(1+\xi)^2 ; N^1 N^2 = \frac{1}{4}(1-\xi^2)
\end{equation}

\begin{equation}
N^1 = \frac{1-\xi}{2} ; N^2 = \frac{1+\xi}{2}
\end{equation}

Then
	\begin{equation}
	\begin{aligned}
	\int N^1 N^1 &= \frac{1}{4} \left[ \xi + \frac{1}{3}\xi^3 \right]_{-1}^{+1} = \frac{2}{3} \\
	\int N^2 N^2 &= \frac{1}{4} \left[ \xi + \frac{1}{3}\xi^3 \right]_{-1}^{+1} = \frac{2}{3} \\
	\int N^1 N^2 &= \frac{1}{4} \left[ \xi - \frac{1}{3}\xi^3 \right]_{-1}^{+1} = \frac{1}{3} \\
	\end{aligned}
	\end{equation}	
	
	\begin{equation}
	\therefore \vec{M}_e = \frac{\lambda \ell}{6} \begin{bmatrix} 2 & 1 \\ 1 & 2 \end{bmatrix}
	\end{equation}
	
	\begin{equation}
	\vec{F}_e^{fA} = \int_0^\ell N f \text{d}x = f \frac{\ell}{2} \int_{-1}^{+1} N \text{d}x = f\frac{\ell}{2} \\
\therefore \vec{F}_e^f = \frac{f\ell}{2} \begin{Bmatrix} 1\\1 \end{Bmatrix}
\end{equation}
 More generally matrix $K_e$ and $M_e$ are: 
\begin{equation}
\begin{aligned}
K_e^{AB} &= \int_0^\ell N_{,x}^A N_{,x}^B \kappa \text{d}x \text{ with } \kappa = N^1 \kappa^1 + N^2 \kappa^2 \\
&= \left(\frac{\ell}{2}\right)\left(\frac{\pm 1}{4}\right)\left(\frac{4}{\ell^e}\right) \int_{-1}^{+1} \left[ \frac{1 -\xi}{2} \kappa^{(1)} + \frac{1 + \xi}{2} \kappa^{(2)} \right] \text{d}\xi\\
\end{aligned}
\end{equation}

and
\begin{equation}
M_e^{AB} = \int_0^\ell N^A N^B \lambda \text{d}x \quad\text{ with }\quad \lambda = N^1 \kappa^{(1)} + N^2 \kappa^{(2)}
\end{equation}

If $A=1$ and $B=1$ then $M_e$ is calculated as shown below in Equation \ref{MeExampleeq1}.
\begin{equation} %%\label{MeExampleeq1}
\begin{aligned}
M_e^{11} &= \left(\frac{\ell}{2}\right) \int_{-1}^{+1} \frac{1}{4} (1-\xi)^2\left[ \frac{1-\xi}{2} \cdot \lambda^{(1)} + \frac{1+\xi}{2} \cdot \lambda^{(2)} \right] \text{d} \xi \\
&= \frac{\ell}{16} \int_{-1}^{+1} \left[
	(1-\xi)^2(1-\xi) \lambda^{(1)}
	+ (1-\xi)^2(1+\xi) \lambda^{(2)}
\right] \text{d}\xi \\
&= \frac{\ell}{16} \left[
	\lambda^{(1)} \int_{-1}^{+1} (1-\xi)^2(1-\xi) \text{d}\xi
	+ \lambda^{(2)} \int_{-1}^{+1} (1-\xi)^2(1+\xi) \text{d}\xi
\right] \\
&= \frac{\ell}{16} \left[
	\lambda^{(1)} \cdot 4
	+ \lambda^{(2)} \cdot \frac{4}{3}
\right] \\
&= \frac{\ell}{4} \left[
	\lambda^{(1)} + \frac{\lambda^{(2)}}{3}
\right]
\end{aligned}
\end{equation}
 
 When $A=2$ and $B=2$:

\begin{equation}%%\label{MeExampleeq2}
\begin{aligned}
M_e^{22} &= \frac{\ell}{16} \int_{-1}^{+1} \left[
	(1+\xi)^2(1-\xi)\lambda^{(1)} + (1+\xi)^2(1+\xi)\lambda^{(2)}
\right] \text{d}\xi \\
&= \frac{\ell}{4} \left[
	\frac{\lambda^{(1)}}{3}
	+ \lambda^{(2)}
\right]
\end{aligned}
\end{equation}
 When $A=1$ and $B=2$ \textbf{or} $A=2$ and $B=1$:
\begin{equation}%%\label{MeExampleeq3}
\begin{aligned}
M_e^{12} &= \frac{\ell}{16} \int_{-1}^{+1} \left[
	(1-\xi^2)(1-\xi) \lambda^{(1)}
	+ (1-\xi^2)(1+\xi)\lambda^{(2)}
\right] \text{d} \xi \\
&= \frac{\ell}{4}\left[
	\frac{\lambda^{(1)}}{3}
	+ \frac{\lambda^{(2)}}{3}
\right] \\
&= \frac{\ell}{12}\left[
	\lambda^{(1)}
	+ \lambda^{(2)}
\right] \\
\end{aligned}
\end{equation}

 The matrix can then be assembled from the above Equations \ref{MeExampleeq1},\ref{MeExampleeq2} and \ref{MeExampleeq3}:
\begin{equation}%%\label{Meassembled}
\begin{aligned}
\vec{M}_e &= \frac{\ell}{4} \left(
	\lambda^{(1)} \begin{bmatrix} 1 & \frac{1}{3} \\ \frac{1}{3} & \frac{1}{3} \end{bmatrix}
	+ \lambda^{(2)} \begin{bmatrix} \frac{1}{3} & \frac{1}{3} \\ \frac{1}{3} & 1 \end{bmatrix}
\right) \\
&= \frac{\ell}{12} \left(
	\lambda^{(1)} \begin{bmatrix} 3 & 1 \\ 1 & 1 \end{bmatrix}
	+ \lambda^{(2)} \begin{bmatrix} 1 & 1 \\ 1 & 3 \end{bmatrix}
\right) \\
\end{aligned}
\end{equation}

The same can be done to assemble the $K$ matrix as shown below.
\begin{equation}%%\label{Keass1}
\begin{aligned}
K_e^{AB} &= \left(\frac{\pm 1}{2\ell}\right) \int_{-1}^{+1} \left[
	\frac{1-\xi}{2} \kappa^{(1)}
	+ \frac{1+\xi}{2} \kappa^{(2)}
\right] \text{d}\xi \\
&= \left(\frac{\pm 1}{2\ell}\right) \left[ \kappa^{(1)} + \kappa^{(2)} \right]
\end{aligned}
\end{equation}


\begin{equation}%%\label{Keass2}
\vec{K}_e = \frac{1}{2\ell} \left(
	\kappa^{(1)} \begin{bmatrix} 1 & -1 \\ -1 & 1 \end{bmatrix}
	+ \kappa^{(2)} \begin{bmatrix} 1 & -1 \\ -1 & 1 \end{bmatrix}
\right)
\end{equation}

There after the $\vec{F}_e$ matrix
\begin{equation}%%\label{Keass3}
\begin{aligned}
F_e^A &= \int_0^\ell N^A f \text{d}x \\
&= \frac{\ell}{2} \int_{-1}^{+1} N^A f \text{d} \xi
\end{aligned}
\end{equation}


\begin{equation}%%\label{Feass4}
\begin{aligned}
F_e^1 &= \frac{\ell}{2} \int_{-1}^{+1} \left(\frac{1-\xi}{2}\right) \cdot \left[
	\frac{1-\xi}{2}\cdot f^{(1)}
	+ \frac{1+\xi}{2}\cdot f^{(2)}
\right] \\
&= \frac{\ell}{8} \int_{-1}^{+1} \left[
	\left(1-\xi\right)^2\cdot f^{(1)}
	+ \left(1-\xi^2\right)\cdot f^{(2)}
\right] \text{d} \xi
\end{aligned}
\end{equation}

\begin{equation}%%\label{Feass5}
\begin{aligned}
F_e^2 &= \frac{\ell}{8} \int_{-1}^{+1} \left[
	(1-\xi^2)\cdot f^{(1)} + (1+\xi)^2\cdot f^{(2)}
\right] \text{d}\xi
\end{aligned}
\end{equation}


\begin{equation}%%\label{Feass6}
\begin{aligned}
\therefore \vec{F}_e &= \frac{\ell}{8} \left(
	f^{(1)} \begin{Bmatrix} \frac{8}{3} \\ \frac{4}{3} \end{Bmatrix}
	+ f^{(2)} \begin{Bmatrix} \frac{4}{3} \\ \frac{8}{3} \end{Bmatrix}
\right)
+ \sum_{A \in \mathcal{A}_N} q^{(A)} \begin{Bmatrix} -\delta^{A1} \\ +\delta^{A2} \end{Bmatrix} \\
&= \frac{\ell}{6} \left(
	f^{(1)} \begin{Bmatrix} 2 \\ 1 \end{Bmatrix}
	+ f^{(2)} \begin{Bmatrix} 1 \\ 2 \end{Bmatrix}
\right)
+ \sum_{A \in \mathcal{A}_N} q^{(A)} \begin{Bmatrix} -\delta^{A1} \\ +\delta^{A2} \end{Bmatrix} \\\\
& \text{TODO:  uppercase or lowercase A1 and A2 with deltas?}
\end{aligned}
\end{equation}


After assembly but before Dirichlet boundaries are applied 
\begin{equation}%%\label{FEM_nodirieq}
\vec{\bar{c}}^T \vec{\bar{K}} \vec{\bar{d}} + \vec{\bar{c}}^T \vec{\bar{M}} \vec{\dot{\bar{d}}} = \vec{\bar{c}}^T \vec{\bar{F}}
\end{equation}
When the Dirichlet boundaries below (Equation \ref{diribound}) are applied to the matrices shown in Equation \ref{FEM_nodirieq} , they can be simplified to Equation \ref{solveeq1}.
\begin{equation}%%\label{diribound}
\begin{aligned}
c_1 = 0 \quad &\text{and} \quad c^{np}=0\\
d_1 = u_{\text{fire(t)}}\quad &\text{and} \quad d^{np} = u_{\text{air}}\\
\end{aligned}
\end{equation}

Then:

\begin{equation}%%\label{solveeq1}
\begin{aligned}
&\vec{c}^T\cdot\vec{K}\cdot\vec{d} + \vec{c}^T\cdot\vec{M}\cdot\vec{\dot{d}} = \vec{c}^T\cdot\vec{F}' - \vec{c}^T\\
&\left(\{K'\}d'+ \{K^{np?}\}d^{np}\right)- \vec{c}^T \left(\{M'\}\dot{d}' + \{M^{np?}\}\dot{d}^{np} \right)
\end{aligned}
\end{equation}


And so:

\begin{equation}\label{appsolveeq2}
\vec{K}\cdot\vec{d} + \vec{M}\cdot\vec{\dot{d}} = \vec{F}' - \vec{F}^{Ke} - \vec{F}^{Me} = \vec{F}
\end{equation}

Solve as

\begin{equation}\label{appsolveeq3}
\begin{aligned}
\vec{\tilde{d}}_{n+1} &= \vec{d}_n + (1-\alpha)\Delta t \vec{v}_n \\
(\vec{M} + \alpha\Delta t\vec{K})\vec{v}_{n+1} &= \vec{F}_{n+1} - \vec{K} \vec{\tilde{d}}_{n+1} \\
\rightarrow \vec{v}_{n+1} &= (\vec{M} + \alpha\Delta t \vec{K})^{-1} (\vec{F}_{n+1} - \vec{K} \vec{\tilde{d}}_{n+1}) \\
\rightarrow \vec{d}_{n+1} &= \vec{\tilde{d}}_{n+1} + \alpha\Delta t \vec{v}_{n+1} \\
\vec{v} &= \vec{\dot{d}}
\end{aligned}
\end{equation}
