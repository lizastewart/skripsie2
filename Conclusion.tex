\chapter{Summary and Conclusion} \label{conclusion}

The usage of Markov Chain Monte-Carlo integration to solve for the thermal diffusivity of cross-laminated SA Pine was successful.
The adaptation of the finite element model derived from the heat conduction and diffusion equation enabled the creation of a posterior distribution.
The posterior distribution modelled after Bayes theory of inverse problems could be traversed and explored using the Markov Chain Monte-Carlo and the Maximum a Posteriori could be found using Nelder-Mead optimization.
The resulting $\kappa$ and $\alpha$ values produced a more accurate model of temperature over time.
There is a lot of potential for further optimization and fine-tuning of these algorithms and models.
With further development this concept could lead to simplified methods of calculating the fire rating of specifically SA-Pine as well as other timber samples.
More accurate and accessible fire ratings will help pave the way to improved fire safety.
