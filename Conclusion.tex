\chapter{Summary and Conclusion} \label{conclusion}

The usage of Markov chain Monte Carlo integration to solve for the thermal diffusivity of cross-laminated SA Pine was successful.
The adaptation of the finite element model derived from the heat conduction and diffusion equation enabled the creation of a posterior distribution.
The posterior distribution modelled after Bayes' theory of inversion could be traversed and explored using the Markov chain Monte Carlo and the maximum a posteriori could be found using Nelder-Mead optimisation.
The resulting $\kappa$ and $\alpha$ values produced a more accurate model of temperature over time.
There is a lot of potential for further optimisation and fine-tuning of these algorithms and models.
With further development, this concept could lead to simplified methods of calculating the fire rating of specifically SA Pine and other timber samples.
More accurate and accessible fire ratings will help pave the way to improved fire safety.
